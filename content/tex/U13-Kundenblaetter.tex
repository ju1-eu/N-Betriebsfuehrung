%ju 17-Jul-22 U13-Kundenblaetter.tex
\textbf{Aufgabe 1)}

\textbf{Wohin wendet sich der Kunde zunächst, wenn er das Autohaus
aufsucht?}

Pkw-Annahme oder Infocenter

\textbf{Aufgabe 2)}

\textbf{Welche Bereiche des Autohauses sind in der Bearbeitung des
Auftrags beteiligt? Nennen Sie die einzelnen Bereiche und erläutern Sie
deren Aufgaben.}

\begin{enumerate}
\item
  \textbf{Kundendienst}

  \begin{itemize}
  \item
    \emph{Aufgaben} Annahme von Reparaturen, technische Beratung des
    Kunden, Fahrzeugübergabe an Kunden, Abwicklung von Garantiefällen
  \item
    \emph{Funktionen} Schnittstelle zwischen Kunden und Werkstatt
  \end{itemize}
\item
  \textbf{Kfz-Werkstatt}

  \begin{itemize}
  \item
    \emph{Aufgaben} Durchführung von Reparaturen und Wartungsarbeiten,
    Einbau von Zubehör
  \item
    \emph{Funktionen} Durchführung der Werkstattarbeiten
  \end{itemize}
\item
  \textbf{Teiledienst}

  \begin{itemize}
  \item
    \emph{Aufgaben} Verwaltung von den Ersatzteilen und Zubehör, Ausgabe
    von Teilen, Verkauf von Teilen
  \item
    \emph{Funktionen} Verwaltung eines Ersatzteile- und
    Zubehör-Sortiment.
  \end{itemize}
\end{enumerate}

\textbf{Aufgabe 3)}

\textbf{Nennen Sie drei weitere Geschäftsbereiche eines Autohauses und
erläutern Sie beispielsweise vier Situationen, in denen ein Kunde mit
dem Bereich Kontakt hat.}

\begin{enumerate}
\item
  \textbf{Geschäftsleitung}

  \begin{itemize}
  \item
    \emph{Aufgaben} Kundenbeschwerde über eine zu hohe Rechnung,
    Betriebsführung, Planung und Organisation
  \item
    \emph{Funktionen} bestimmt Geschäftspolitik und legt die Zielsetzung
    des Autohauses fest
  \end{itemize}
\item
  \textbf{Verkauf}

  \begin{itemize}
  \item
    \emph{Aufgaben} Kundenberatung, Neuwagenverkauf, Verkauf von
    Gebrauchtwagen, Fahrzeugauslieferung und -übergabe, Bewertung von
    Gebrauchtwagen
  \item
    \emph{Funktionen} Umsatz von Fahrzeugen
  \end{itemize}
\item
  \textbf{Verwaltung}

  \begin{itemize}
  \item
    \emph{Aufgaben} Zahlungserinnerung einer nicht gezahlten Rechnung an
    den Kunden, Buchhaltung, Abwicklung von Geschäften mit Lieferanten
    und Herstellern, Lohn- und Gehaltsabrechnung
  \item
    \emph{Funktionen} kaufmännische Aufgaben
  \end{itemize}
\end{enumerate}

\textbf{Aufgabe 4)}

\textbf{Beschreiben Sie die Vorgehensweise des Kundendienstberaters bei
der Auftragsannahme.}

\begin{enumerate}
\item
  Fragen nach dem Kundenwunsch
\item
  Durchführung der Untersuchung des Fahrzeuges
\item
  Dokumentation von Schäden am Fahrzeug
\item
  Erfassung von Wertgegenständen im Fahrzeug
\item
  Probefahrt mit dem Kunden
\item
  Mitteilung des kalkulierten Preises
\item
  Auftrag erstellen
\end{enumerate}

\textbf{Aufgabe 5)}

\textbf{Die Kundenzufriedenheit wird im Wesentlichen durch die
technische Produktqualität und die Servicequalität beeinflusst. Nennen
Sie die Merkmale.}

\begin{enumerate}
\item
  \textbf{Technische Produktqualität}

  \begin{itemize}
  \item
    Verarbeitung und Reparaturanfälligkeit
  \item
    Ausführung von Wartungs- und Reparaturarbeiten
  \end{itemize}
\item
  \textbf{Servicequalität}

  \begin{itemize}
  \item
    Kulanzregelungen
  \item
    Einhaltung von Terminen
  \item
    Qualität der Beratung
  \item
    Umgang mit Reklamationen
  \end{itemize}
\end{enumerate}

\textbf{Aufgabe 6)}

\textbf{Nennen Sie Beispiele für Servicekonzepte, mit denen das Autohaus
die Kundenbindung verbessern kann.}

\begin{itemize}
\item
  Werbung
\item
  Garantie und Kulanz
\item
  Hol- und Bring-Service
\item
  Reparatur-Finanzierung
\item
  Dienstleistungsangebote: Verkauf, Wartung
\end{itemize}

\textbf{Aufgabe 7)}

\textbf{Erläutern Sie die wesentlichen Merkmale der jeweiligen
Kundenarten und geben Sie die Bedeutung für den Betrieb an. Beschreiben
Sie die Erwartung des Kunden an den Betrieb und notwendige Maßnahmen des
Betriebes.}

Kundenarten: Laufkunde, Dauerkunde, Stammkunde, Großkunde

\begin{enumerate}
\item
  \textbf{Laufkunde} (Kommt zufällig und hat keine Bindung)

  \begin{itemize}
  \item
    \emph{Bedeutung} Gering
  \item
    \emph{Erwartung des Kunden} Schnelle und zuverlässige Ausführung der
    Arbeit
  \item
    \emph{Maßnahmen} Keine
  \end{itemize}
\item
  \textbf{Dauerkunde} (nimmt gelegentlich Service in Anspruch)

  \begin{itemize}
  \item
    \emph{Bedeutung} Mittel
  \item
    \emph{Erwartung des Kunden} zuverlässig und preisgünstig
  \item
    \emph{Maßnahmen} Angebote an Kunden
  \end{itemize}
\item
  \textbf{Stammkunde} (lässt alle Arbeiten in der Werkstatt ausführen)

  \begin{itemize}
  \item
    \emph{Bedeutung} Hoch, Wachstum und Gewinn kann erwartet werden,
    Weiterempfehlung des Betriebes
  \item
    \emph{Erwartung des Kunden} persönliche Betreuung
  \item
    \emph{Maßnahmen} persönliche Ansprache
  \end{itemize}
\item
  \textbf{Großkunde} (Gesamten Fuhrpark warten)

  \begin{itemize}
  \item
    \emph{Bedeutung} sehr hoch
  \item
    \emph{Erwartung des Kunden} Schnelle und gute Ausführung, Kulanz
  \item
    \emph{Maßnahmen} Rabatt, Terminvereinbarung
  \end{itemize}
\end{enumerate}
