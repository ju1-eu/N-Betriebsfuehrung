%ju 06-Jun-22 U05-Gesamtarbeitszeit-StVs-AW-Vs-Flh-Loesung.tex
\textbf{Ü5 - Gesamtarbeitszeit - St-Vs - AW-VS - Produktivität}

\textbf{Aufgabe 1)}

\begin{enumerate}
\def\labelenumi{\alph{enumi})}
\item
  \textbf{Gesamtarbeitszeit}
\end{enumerate}

\begin{itemize}
\item
  3 x 157,5 h x 12 = 5.670 h/Jahr
\end{itemize}

\begin{enumerate}
\def\labelenumi{\alph{enumi})}
\setcounter{enumi}{1}
\item
  \textbf{Fertigungslohnstunden} 75 \%
\end{enumerate}

\begin{itemize}
\item
  5.670 x 0,75 = 4.252,5 h/Jahr
\end{itemize}

\begin{enumerate}
\def\labelenumi{\alph{enumi})}
\setcounter{enumi}{2}
\item
  \textbf{Fertigungslohnstunden}
\end{enumerate}

\begin{itemize}
\item
  4.252,5 / 12 = 354,38 h/Monat
\end{itemize}

\begin{enumerate}
\def\labelenumi{\alph{enumi})}
\setcounter{enumi}{3}
\item
  \textbf{Hilfslohnstunden} 25 \%
\end{enumerate}

\begin{itemize}
\item
  5.670 x 0,25 / 12 = 118,13 h/Monat
\end{itemize}

\begin{enumerate}
\def\labelenumi{\alph{enumi})}
\setcounter{enumi}{4}
\item
  \textbf{Lohn eines Monteurs im Monat}
\end{enumerate}

\begin{itemize}
\item
  157,5 h/Monat x 12,00 €/h = 1.890 €/Monat
\end{itemize}

\begin{enumerate}
\def\labelenumi{\alph{enumi})}
\setcounter{enumi}{5}
\item
  \textbf{Fertigungslohn eines Monteurs im Monat}
\end{enumerate}

\begin{itemize}
\item
  1.890 €/Monat x 0,75 = 1.417,5 €/Monat
\end{itemize}

\textbf{Aufgabe 2)}

\begin{enumerate}
\def\labelenumi{\alph{enumi})}
\item
  \textbf{AW-Lohnsatz} (Monteur)
\end{enumerate}

\begin{itemize}
\item
  AW-ls = SLs / WF
\item
  AW-ls = 12,20 €/h / 12 AW/h = 1,02 €/AW
\end{itemize}

\begin{enumerate}
\def\labelenumi{\alph{enumi})}
\setcounter{enumi}{1}
\item
  \textbf{AW-Verrechnungssatz} (Kunde)
\end{enumerate}

\begin{itemize}
\item
  AW-VS = St-Vs/WF
\item
  AW-VS = 52 €/h / 12 AW/h = 4,33 €/AW
\end{itemize}

\begin{enumerate}
\def\labelenumi{\alph{enumi})}
\setcounter{enumi}{2}
\item
  \textbf{Stundenverrechnungssatz brutto}
\end{enumerate}

\begin{itemize}
\item
  St-Vs = 52 €/h + 9,88 (19 \%) = 61,88 €/h

  \begin{itemize}
  \item
    \textbf{Alternative}
  \item
    52 €/h x 1,19 = 61,88 €/h
  \end{itemize}
\end{itemize}

\textbf{Aufgabe 3)}

\begin{enumerate}
\def\labelenumi{\alph{enumi})}
\item
  \textbf{Fertigungslohnstunden}
\end{enumerate}

\begin{itemize}
\item
  160 h x 75 \% = 120 h/Monat
\end{itemize}

\begin{enumerate}
\def\labelenumi{\alph{enumi})}
\setcounter{enumi}{1}
\item
  \textbf{Soll-Leistung in AW}
\end{enumerate}

\begin{itemize}
\item
  Soll-AW = Flh x WF
\item
  Soll-AW = 120 h/Monat x 12 AW/h = 1.440 AW/h
\end{itemize}

\begin{enumerate}
\def\labelenumi{\alph{enumi})}
\setcounter{enumi}{2}
\item
  \textbf{Leistungsgrad}
\end{enumerate}

\begin{itemize}
\item
  LG = Ist-AW / Soll-AW
\item
  LG = 1.656 AW / 1440 AW = 1,15

  \begin{itemize}
  \item
    (1,15 $\to$ hat 15 /\% mehr gemacht)
  \end{itemize}
\end{itemize}

\begin{enumerate}
\def\labelenumi{\alph{enumi})}
\setcounter{enumi}{3}
\item
  \textbf{Leistungslohnsatz}
\end{enumerate}

\begin{itemize}
\item
  LLS = Stundenlohnsatz x Leistungsgrad
\item
  LLS = 11,80 €/h x 1,15 = 13,57 €/h
\end{itemize}

\begin{enumerate}
\def\labelenumi{\alph{enumi})}
\setcounter{enumi}{4}
\item
  \textbf{Fertigungslohn}
\end{enumerate}

\begin{itemize}
\item
  FL = 120 h x 13,57 €/h = 1.628,40 €
\end{itemize}

\begin{enumerate}
\def\labelenumi{\alph{enumi})}
\setcounter{enumi}{5}
\item
  \textbf{Hilfslohn}
\end{enumerate}

\begin{itemize}
\item
  HL = 40 h x 11,80 €/h = 472 €

  \begin{itemize}
  \item
    160 h

    \begin{itemize}
    \item
      $\to$ 75 \% = 120 h und
    \item
      $\to$ 25 \% = 40 h
    \end{itemize}
  \end{itemize}
\end{itemize}

\begin{enumerate}
\def\labelenumi{\alph{enumi})}
\setcounter{enumi}{6}
\item
  \textbf{Lohn}
\end{enumerate}

\begin{itemize}
\item
  Lohn = Fertigungslohn + Hilfslohn
\item
  Lohn = 1.628,40 € + 472 € = 2.100,40 €
\end{itemize}

\begin{enumerate}
\def\labelenumi{\alph{enumi})}
\setcounter{enumi}{7}
\item
  \textbf{AW-Verrechnungssatz}
\end{enumerate}

\begin{itemize}
\item
  AW-VS = St-Vs / WF
\item
  AW-VS = 52,92 €/h / 12 AW/h = 4,41 €/AW
\end{itemize}

\begin{enumerate}
\def\labelenumi{\roman{enumi})}
\item
  \textbf{Lohnerlös}
\end{enumerate}

\begin{itemize}
\item
  LE = Ist-AW x AW-VS
\item
  LE = 1.656 AW x 4,41 €/AW = 7.302,96 €
\end{itemize}

\textbf{Aufgabe 4)}

\begin{enumerate}
\def\labelenumi{\alph{enumi})}
\item
  \textbf{Fertigungslohnstunden je Monteur und Jahr}
\end{enumerate}

\begin{itemize}
\item
  183,5 Tage x 7,5 h/Tag = 1.376,25 h/Jahr

  \begin{itemize}
  \item
    265 T x 0,10 (10 \%) = 26,5 Tage
  \item
    30 U + 17 K + 8 F = 55 Tage
  \item
    265 T - 55 T - 26,5 T = 183,5 Tage
  \end{itemize}
\end{itemize}

\begin{enumerate}
\def\labelenumi{\alph{enumi})}
\setcounter{enumi}{1}
\item
  \textbf{Anteil der Fertigungslohnstunden in \% der Gesamtarbeitszeit}
\end{enumerate}

\begin{itemize}
\item
  Flh = Flh x 100 \% / Arbeitszeit (komplett)
\item
  Flh = 1.376,25 h/Jahr x 100 \% / 1.987,5 h = 69,25 \% (Produktiv)

  \begin{itemize}
  \item
    NR) 265 T x 7,5 h = 1.987,5 h
  \item
    \textbf{Alternative} $\to$

    \begin{itemize}
    \item
      Anwesenheitstage x 100 \% / mögliche Arbeitstage
    \item
      = 183,5 T x 100 \% / 265 T = 69,25 \%
    \end{itemize}
  \end{itemize}
\end{itemize}

\begin{enumerate}
\def\labelenumi{\alph{enumi})}
\setcounter{enumi}{2}
\item
  \textbf{Produktivität} (in \%)
\end{enumerate}

\begin{itemize}
\item
  = Flh x 100 \% / AZ
\item
  = 1.376,25 h/Jahr x 100 \% / 1.987,5 h = 69,25 \%
\end{itemize}

\textbf{Aufgabe 5)}

\begin{enumerate}
\def\labelenumi{\alph{enumi})}
\item
  \textbf{Gewinn}
\end{enumerate}

\begin{itemize}
\item
  Gewinn (in €) = UE - EK - GK
\item
  Gewinn (in €) = 460.940 - 131.264 - 288.616 = 41.060,00 €

  \begin{itemize}
  \item
    Lohn gesamt = 164.080

    \begin{itemize}
    \item
      $\to$ 80 \% 131.264 (EK, Fertigungslohn) und
    \item
      $\to$ 20 \% 32.816 (Hilfslohn)
    \end{itemize}
  \item
    GK = Restgemeinkosten + Hilfslohn
  \item
    GK = 255.800 + 32.816 = 288.616 €
  \item
    Seko = EK + GK
  \item
    Seko = 131.264 + 288.616 = 419.880 €
  \end{itemize}
\item
  Gewinn (in \%) = Gewinn x 100 \% / Seko
\item
  Gewinn (in \%) = 41.060 x 100 \% / 419.880 = 9,78 \%
\end{itemize}

\begin{enumerate}
\def\labelenumi{\alph{enumi})}
\setcounter{enumi}{1}
\item
  \textbf{Erlösindex}
\end{enumerate}

\begin{itemize}
\item
  EL = LE / FL
\item
  EL = 460.940 / 131.264 = 3,51
\end{itemize}

\begin{enumerate}
\def\labelenumi{\alph{enumi})}
\setcounter{enumi}{2}
\item
  erlösten \textbf{Stundenverrechnungssatz}
\end{enumerate}

\begin{itemize}
\item
  St-Vs = LE / Flh
\item
  St-Vs = 460.940 / 10.110 = 45,59 €/h
\end{itemize}

\begin{enumerate}
\def\labelenumi{\alph{enumi})}
\setcounter{enumi}{3}
\item
  erlösten \textbf{AW-Verrechnungssatz}
\end{enumerate}

\begin{itemize}
\item
  AW-VS = St-Vs / WF
\item
  AW-VS = 45,59 €/h / 12 AW/h = 3,80 €/AW
\end{itemize}
