%ju 17-Jul-22 U18-Simulation1-Loesung.tex
\textbf{Aufgabe 3)}

\textbf{Was verstehen Sie unter dem Begriff kundenorientiertes
Qualitätsmanagement?}

Die Zufriedenheit des Kunden spiegelt die Qualität des Produktes bzw.
der Leistung wider. Deshalb sollte man die Kundenansprüche genau kennen
und in den Vordergrund des QM-Systems stellen.

\textbf{Aufgabe 4)}

\textbf{Nennen Sie 5 Gesichtspunkte für die Entscheidung, ob Sie einen
Generator überholen oder ihn durch ein AT-Teil austauschen.}

\begin{enumerate}
\item
  Preisunterschied zwischen Instandsetzung und Austauschteil. Was ist
  preiswerter?
\item
  Ist eine Zeitwert gerechte Instandsetzung möglich?
\item
  Art des Schadens feststellen: Verschleißteile defekt oder
  Ständerwicklung?
\item
  Teileverfügbarkeit prüfen
\item
  Was möchte der Kunde?
\item
  Instandsetzungsfähigkeit des Bauteils feststellen
\item
  Ist noch Garantie vorhanden?
\end{enumerate}

Ersatzteile \footnote{\url{https://hc-cargo.de/}}

\textbf{Aufgabe 7)}

\textbf{Zählen Sie fünf Möglichkeiten zur Ermittlung des
Reparaturaufwandes auf}

\begin{enumerate}
\item
  Kostenvoranschlag
\item
  Dialogannahme
\item
  Gutachten
\item
  Wartungsplan
\item
  AW-Vorgabezeiten
\item
  Herstellervorgaben
\end{enumerate}

\textbf{Aufgabe 8)}

\begin{enumerate}
\item
  \textbf{Was versteht man unter Fertigungslöhnen?}
\end{enumerate}

Löhne für Kundenaufträge (K-Aufträge)

\emph{Beispiel:}

\begin{itemize}
\item
  Reparaturlöhne an Kundenfahrzeugen
\item
  produktive Anteile vom Lehrlingslohn.
\end{itemize}

\begin{enumerate}
\setcounter{enumi}{1}
\item
  \textbf{Was versteht man unter Hilfslöhnen?}
\end{enumerate}

Löhne für unproduktive Stunden, die nicht unmittelbar mit der Fertigung
bzw. Reparatur zusammenhängen und von der Werkstatt getragen werden
müssen.

\begin{enumerate}
\setcounter{enumi}{2}
\item
  \textbf{Woraus setzen sich Hilfslöhne zusammen?}
\end{enumerate}

\begin{itemize}
\item
  allgemeine Werkstattarbeiten
\item
  Nacharbeiten, Gewährleistungen und Kulanzarbeiten, die von der
  Werkstatt getragen werden müssen
\item
  Leerlauf- und Wartezeiten
\item
  Wartung und Reparatur von firmeneigenen Fahrzeugen
\item
  Ausbildungsvergütungen
\item
  Urlaub, Feiertage
\item
  Tarifliches Urlaubsgeld
\item
  Lohnfortzahlung im Krankheitsfall
\end{itemize}

\begin{enumerate}
\setcounter{enumi}{3}
\item
  \textbf{Was versteht man unter Einzelkosten?}
\end{enumerate}

Einzelkosten sind unmittelbar auf eine betriebliche Leistung bezogen und
können direkt zugeordnet werden.

\emph{Beispiel:}

\begin{itemize}
\item
  Produktive Fertigungslöhne
\item
  Fertigungsmaterialien (Ersatzteile).
\end{itemize}

\begin{enumerate}
\setcounter{enumi}{4}
\item
  \textbf{Was sind Gemeinkosten?}
\end{enumerate}

Kosten, die einzeln nicht ermittelt werden können, da sie sich auf alle
betrieblichen Leistungen aufteilen. Sie müssen deshalb aus allen
Kostenstellen erfasst werden.

\begin{enumerate}
\setcounter{enumi}{5}
\item
  \textbf{Woraus setzen sich die Gemeinkosten eines Betriebes zusammen?}
\end{enumerate}

\begin{itemize}
\item
  Hilfslöhne
\item
  Löhne und Gehälter für indirekt produktives Personal (unproduktiv)
\item
  soziale Aufwendungen
\item
  Raumkosten
\item
  Abschreibungen
\item
  Instandhaltung
\item
  Hilfs- und Betriebsstoffe (Materialgemeinkosten)
\item
  betriebliche Steuern
\item
  Versicherungsbeiträge
\item
  Gebühren
\item
  verschiedene Kosten (Werbekosten, Reisekosten)
\end{itemize}

\begin{enumerate}
\setcounter{enumi}{6}
\item
  \textbf{Was sind kalkulatorische Kosten?}
\end{enumerate}

Kalkulatorische Kosten sind aufwandsfremde Kosten, d.h. der durch sie
erfasste Werteverbrauch steht in der Buchhaltung überhaupt nicht oder in
andere Form.

Die kalkulatorischen Kosten erfassen den betriebsbedingten Aufwand, der
in der Aufwandsrechnung gar nicht oder in einer anderen Form und Höhe
ausgewiesen ist, die für die Kostenrechnung ungeeignet ist.

Den kalkulatorischen Kosten liegen keine Rechnungen zugrunde, deshalb
müssen sie kalkulatorisch berücksichtigt werden. Je nachdem, ob es sich
um Kosten handelt, die in der Finanzbuchführung -- wenn auch in anderer
Höhe -- als Aufwand erfasst werden oder ob es sich um Kosten handelt,
die gar nicht als Aufwand erfasst sind (bzw. erfasst werden dürfen),
spricht man auch von Anderskosten bzw. Zusatzkosten.

\emph{Beispiele:}

\begin{itemize}
\item
  kalkulatorische Unternehmerlohn
\item
  kalkulatorische Abschreibungen
\item
  kalkulatorische Miete
\item
  kalkulatorische Wagnisse
\item
  kalkulatorische Zinsen
\item
  kalkulatorischer Gewinn
\end{itemize}

\begin{enumerate}
\setcounter{enumi}{7}
\item
  \textbf{Wie werden die Kosten für W-Aufträge verrechnet?}
\end{enumerate}

Sie werden den Gemeinkosten zugeschlagen, da die Werkstatt keine
direkten Erlöse für W-Aufträge bekommt. Die dabei entstandenen
Lohnkosten nennt man Hilfslöhne.

\emph{Beispiele:}

\begin{itemize}
\item
  Urlaubs- und Fertigungslöhne
\item
  Lohnfortzahlungen im Krankheitsfall
\item
  sonstige bezahlte Arbeitsversäumnisse
\end{itemize}

\begin{enumerate}
\setcounter{enumi}{8}
\item
  \textbf{Was versteht man unter dem Kostenindex?}
\end{enumerate}

Kennzahl für die Vorkalkulation (Werkstattindex). Er gibt an, wie viel
mal mehr der Kunde für eine Fertigungslohnstunde zu bezahlen hat, als
der Monteur in dieser Stunde verdient.

\begin{enumerate}
\setcounter{enumi}{9}
\item
  \textbf{Wie wird der Kostenindex ermittelt?}
\end{enumerate}

Die Summe aus produktive Fertigungslöhnen, Gemeinkosten und Gewinn
dividiert durch die produktive Fertigungslöhne ergibt den Kostenindex.

$\boxed{\text{KI} = \frac{\text{produktive Fertigungslöhne} + \text{Gemeinkosten} + \text{Gewinn}}{\text{produktive Fertigungslöhne}}}$

\begin{enumerate}
\setcounter{enumi}{10}
\item
  \textbf{Was versteht man unter Wirtschaftlichkeit?}
\end{enumerate}

Die Wirtschaftlichkeit ist der Quotient aus den Erlösen und
Selbstkosten.

Ist die Wirtschaftlichkeit größer als eins, so ist ein Gewinn erzielt
worden. Die zwei Stellen nach dem Komma geben den prozentualen Gewinn,
bezogen auf die Selbstkosten, an. Ist dieser höher als kalkuliert, so
ist mehr Gewinn erzielt worden, als geplant wurde.

$\boxed{\text{Wirtschaftlichkeit} = \frac{\text{Umsatzerlöse}}{\text{Selbstkosten}}}$

\emph{Beispiel:}
$\text{WI} = 2,05~\% \quad 2 > 1 \to \text{ Gewinn und } 5~\% \text{ mehr als geplant}$

\begin{enumerate}
\setcounter{enumi}{11}
\item
  \textbf{Was sind Lohnerlöse (Werkstattumsatz)?}
\end{enumerate}

Lohnerlöse sind die Summe aus den produktiven Fertigungslöhnen, den
Gemeinkosten und dem Gewinn, innerhalb einer Abrechnungsperiode, ohne
dass die Materialkosten berücksichtigt sind. Lohnerlöse sind auch Erlöse
aus K-, I- und G-Aufträgen (Gewährleistungsaufträge) ohne Material.

$\boxed{\text{Lohnerlöse} = \text{produktiven Fertigungslöhnen} + \text{Gemeinkosten} + \text{Gewinn}}$

\begin{enumerate}
\setcounter{enumi}{12}
\item
  \textbf{Was sind Arbeitswerte?}
\end{enumerate}

Richtzeiten, die für alle normalen Wartungs- und Instandhaltungsarbeiten
vom Hersteller festgelegt werden.

\begin{enumerate}
\setcounter{enumi}{13}
\item
  \textbf{Wie werden Arbeitswerte festgelegt?}
\end{enumerate}

Sie werden für jede Arbeit so bemessen, durch Zeitstudien von REFA, dass
sie für jeden Mechaniker eine Mindestleistung darstellen. Rüst- und
Verteilzeiten, wie abholen von Ersatzteilen und Sonderwerkzeugen aus dem
Lager, sind mit berücksichtigt.

\begin{enumerate}
\setcounter{enumi}{14}
\item
  \textbf{Erklären Sie den Begriff Soll-Leistung.}
\end{enumerate}

Die SOLL-Leistung ist die Mindestleistung, die ein im Leistungslohn
arbeitender Mechaniker pro Stunde erreichen soll (Werkstattfaktor,
Normalleistung).

\begin{enumerate}
\setcounter{enumi}{15}
\item
  \textbf{Erklären Sie den Begriff Ist-Leistung.}
\end{enumerate}

Die IST-Leistung ist die tatsächlich erbrachte Leistung.

\begin{enumerate}
\setcounter{enumi}{16}
\item
  \textbf{Welche Aussage macht der Leistungsgrad bei Arbeiten im
  Leistungslohn?}
\end{enumerate}

Er gibt das Verhältnis von IST-Leistung zu SOLL-Leistung an.

$\boxed{\text{LG} = \frac{\text{Ist-AW}}{\text{Soll-AW}}}$

\begin{enumerate}
\setcounter{enumi}{17}
\item
  \textbf{Wozu dient der AW-Verrechnungssatz?}
\end{enumerate}

Bei Werkstätten, die im Leistungslohn arbeiten, dient er zur ermittlung
des Arbeitspreises für eine Arbeitsposition

$\boxed{\text{AW-Vs} = \frac{\text{StVs}}{\text{WF}}}$

\begin{enumerate}
\setcounter{enumi}{18}
\item
  \textbf{Woraus errechnet sich der Leistungslohn?}
\end{enumerate}

Anzahl der erreichten Arbeitswerte multipliziert mit der AW-Vergütung
oder aus dem garantierten Grundlohn plus Leistungszulage.
