%ju 17-Jul-22 U10-Pruefungsaufgabentraining-A6-Loesung.tex
\textbf{Aufgabe 2)}

\textbf{Erstellen Sie eine Arbeitsplanung, in der sie stichwortartig den
Ablauf von der Auftragsannahme bis zur Fahrzeugrückgabe an den Kunden in
möglichst kleinen Schritten aufführen.}

\begin{enumerate}
\item
  \textbf{Terminvereinbarung} Auftragsannahme

  \begin{itemize}
  \item
    Termin mit Kunden vereinbaren
  \end{itemize}
\item
  \textbf{Terminvorbereitung}

  \begin{itemize}
  \item
    KD-Berater plant Fahrzeugdurchsicht auf Basis Fahrzeughistorie
  \end{itemize}
\item
  \textbf{Fahrzeugannahme}

  \begin{itemize}
  \item
    Fahrzeug wird vom KD-Berater übernommen und Fahrzeugcheck
    durchgeführt
  \end{itemize}
\item
  \textbf{Auftragserstellung}

  \begin{itemize}
  \item
    notwendige Arbeiten erfassen und Werkstattauftrag erstellen
  \item
    Teileverfügbarkeit prüfen
  \end{itemize}
\item
  \textbf{Reparatur}

  \begin{itemize}
  \item
    In der Werkstatt wird nach Herstellervorgaben des Fahrzeug instand
    gesetzt
  \end{itemize}
\item
  \textbf{Qualitätskontrolle}

  \begin{itemize}
  \item
    Ausführung der Arbeit überprüfen, Endkontrolle / Sichtkontrolle /
    Probefahrt
  \end{itemize}
\item
  \textbf{Vorbereiten der Fahrzeugrückgabe}

  \begin{itemize}
  \item
    Rückgabe vorbereiten und Rechnung erstellen, Rechnung prüfen
  \end{itemize}
\item
  \textbf{Fahrzeugrückgabe}

  \begin{itemize}
  \item
    Fahrzeug an Kunde übergeben und Arbeiten anhand der Rechnung
    erläutern, Kunde zahlt Rechnung
  \end{itemize}
\item
  \textbf{Nachbearbeitung}

  \begin{itemize}
  \item
    Kundenzufriedenheit prüfen anhand von Nachfragen
  \item
    anonymer Fragebogen (telefonisch, Internet, Post)
  \end{itemize}
\end{enumerate}

\textbf{Aufgabe 6)}

\textbf{1. Welche Bedeutung hat der Kunde für Kfz-Betrieb? Machen Sie
drei Angaben.}

\begin{itemize}
\item
  Auftraggeber
\item
  Geldgeber
\item
  Kunde kann ein positiver Multiplikator sein
\item
  indirekter Arbeitgeber
\end{itemize}

\textbf{2. Welche Gründe bewegen einen Kunden, das Autohaus zu wechseln?
Nennen Sie drei.}

\begin{itemize}
\item
  Unzufriedenheit wegen Preis/Leistung
\item
  Termintreue
\item
  Markenwechsel
\item
  Umgang mit dem Kunden
\end{itemize}

\textbf{3. Begründen Sie die möglichen unterschiedlichen Kriterien für
den Kauf eines Autos zwischen einer Familie und einer Einzelperson.
Nennen Sie jeweils drei Gesichtspunkte.}

\textbf{Kauf eines Autos für die Einzelperson:} Kleinwagen, Sportwagen,
Musikanlage, Tuning

\textbf{Kauf eines Autos für die Familie:} Großraumwagen, Platzbedarf,
Verbrauch, Anschaffungskosten, Sicherheit

\textbf{4. Nennen Sie drei Grundregeln für den ersten Kontakt mit einem
Kunden.}

\begin{itemize}
\item
  Vertrauen aufbauen -- Small Talk
\item
  Erster Eindruck -- Kleidung/Körpersprache -- gepflegtes Äußeres
\item
  Gespräch mit offener Frage beginnen: Wie kann ich Ihnen helfen?
\item
  Aktives Zuhören
\end{itemize}

\textbf{5. Wie gehen Sie mit einem Kunden um, der sich über eine
Reparatur beschwert?}

\begin{itemize}
\item
  Kleinigkeiten sofort erledigen
\item
  Richtig Entschuldigen bei eigenes Verschulden
\item
  Beschwerde ernst nehmen
\item
  Zeit nehmen, ausreden lassen
\end{itemize}

\textbf{6. Nennen Sie vier Möglichkeiten, wie die Kundenzufriedenheit
gefördert werden kann.}

\begin{itemize}
\item
  Fachgerechte Reparatur
\item
  Termin einhalten, guter Service
\item
  gutes Preis-Leistungs-Verhältnis
\item
  freundliches Auftreten
\end{itemize}
