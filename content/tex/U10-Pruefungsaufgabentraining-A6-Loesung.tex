%ju 06-Jun-22 U10-Pruefungsaufgabentraining-A6-Loesung.tex
\textbf{Ü10 - Prüfungsaufgabentraining Aufgabe 6}

\textbf{1. Welche Bedeutung hat der Kunde für Kfz-Betrieb?}

\begin{itemize}
\item
  Auftraggeber
\item
  Geldgeber
\item
  Kunde kann ein positiver Multiplikator sein
\item
  indirekter Arbeitgeber
\end{itemize}

\textbf{2. Wann hauen die Kunden ab? Nenne Gründe (Autohauswechsel)}

\begin{itemize}
\item
  Unzufriedenheit wegen Preis/Leistung
\item
  Termintreue
\item
  Markenwechsel
\item
  Umgang mit dem Kunden
\end{itemize}

\textbf{3. Kriterien/Ansprüche für den Kauf eines Autos}

\begin{table}[!ht]% hier: !ht 
\centering 
	\caption{}% \label{tab:}%% anpassen 
\begin{tabular}{@{}ll@{}}
\hline
\textbf{Einzelperson} & \textbf{Familie} \\
\hline
Kleinwagen & Großraumwagen \\
Sportwagen & Platzbedarf \\
Musikanlage & Verbrauch \\
Tuning & Anschaffungskosten \\
- & Sicherheit \\
\hline
\end{tabular} 
\end{table}

\textbf{4. Grundregeln für den ersten Kundenkontakt}

\begin{itemize}
\item
  Vertrauen aufbauen -- Small Talk
\item
  Erster Eindruck -- Kleidung/Körpersprache -- gepflegtes Äußeres
\item
  Gespräch mit offener Frage beginnen: Wie kann ich Ihnen helfen?
\item
  Aktives Zuhören
\end{itemize}

\textbf{5. Umgang mit Kundenbeschwerde über Reparatur}

\begin{itemize}
\item
  Kleinigkeiten sofort erledigen
\item
  Richtig Entschuldigen bei eigenes Verschulden
\item
  Beschwerde ernst nehmen
\item
  Zeit nehmen, ausreden lassen
\end{itemize}

\textbf{6. Möglichkeiten, um Kundenzufriedenheit zu fördern}

\begin{itemize}
\item
  Fachgerechte Reparatur
\item
  Termin einhalten, guter Service
\item
  gutes Preis-Leistungs-Verhältnis
\item
  freundliches Auftreten
\end{itemize}
