%ju 28-Mai-22 03-Betriebsfuehrung.tex
\section{Betriebsorganisation}\label{betriebsorganisation}

Ein Unternehmen ist auf die Optimierung des Gewinns ausgerichtet. Dies
wird erreicht durch den optimalen Einsatz von Mitarbeitern, Maschinen,
Material und Zeit.

\newpage

\subsection{Aufbauorganisation -- Geschäftsbereiche eines
Autohauses}\label{aufbauorganisation-geschaeftsbereiche-eines-autohauses}

\textbf{Organigramm} $\to$ Hierarchisch strukturiert,
Organisationsstruktur, Weisungsbeziehungen

\textbf{Softskills} $\to$ Selbstsicherheit, Selbstständigkeit,
Entscheidungsfähigkeit

\begin{enumerate}
\item
  \textbf{Geschäftsleitung}

  \begin{itemize}
  \item
    \emph{Aufgaben} Kundenbeschwerde über eine zu hohe Rechnung,
    Betriebsführung, Planung und Organisation
  \item
    \emph{Funktionen} bestimmt Geschäftspolitik und legt die Zielsetzung
    des Autohauses fest
  \end{itemize}
\item
  \textbf{Kundendienst}

  \begin{itemize}
  \item
    \emph{Aufgaben} Annahme von Reparaturen, technische Beratung des
    Kunden, Fahrzeugübergabe an Kunden, Abwicklung von Garantiefällen
  \item
    \emph{Funktionen} Schnittstelle zwischen Kunden und Werkstatt
  \end{itemize}
\item
  \textbf{Kfz-Werkstatt}

  \begin{itemize}
  \item
    \emph{Aufgaben} Durchführung von Reparaturen und Wartungsarbeiten,
    Einbau von Zubehör
  \item
    \emph{Funktionen} Durchführung der Werkstattarbeiten
  \end{itemize}
\item
  \textbf{Teiledienst}

  \begin{itemize}
  \item
    \emph{Aufgaben} Verwaltung von den Ersatzteilen und Zubehör, Ausgabe
    von Teilen, Verkauf von Teilen
  \item
    \emph{Funktionen} Verwaltung eines Ersatzteile- und
    Zubehörsortiments
  \end{itemize}
\item
  \textbf{Verkauf}

  \begin{itemize}
  \item
    \emph{Aufgaben} Kundenberatung, Neuwagenverkauf, Verkauf von
    Gebrauchtwagen, Fahrzeugauslieferung und -übergabe, Bewertung von
    Gebrauchtwagen
  \item
    \emph{Funktionen} Umsatz von Fahrzeugen
  \end{itemize}
\item
  \textbf{Verwaltung}

  \begin{itemize}
  \item
    \emph{Aufgaben} Zahlungserinnerung einer nicht gezahlten Rechnung an
    den Kunden, Buchhaltung, Abwicklung von Geschäften mit Lieferanten
    und Herstellern, Lohn- und Gehaltsabrechnung
  \item
    \emph{Funktionen} kaufmännische Aufgaben
  \end{itemize}
\end{enumerate}

\newpage

\subsection{Kunden und Betrieb}\label{kunden-und-betrieb}

\textbf{Kundenorientierung} ist die Ausrichtung des Denkens und Handelns
der Mitarbeiter auf den Kunden und seine Bedürfnisse. Macht das
wirtschaftlich Sinn? Kundenanforderungen zu erfüllen oder Erwartungen
des Kunden zu übertreffen.

\textbf{Was beeinflusst die Kundenzufriedenheit? Nenne Merkmale}

\begin{enumerate}
\item
  \textbf{Technische Produktqualität}

  \begin{itemize}
  \item
    Verarbeitung und Reparaturanfälligkeit
  \item
    Ausführung von Wartungs- und Reparaturarbeiten
  \end{itemize}
\item
  \textbf{Servicequalität}

  \begin{itemize}
  \item
    Kulanzregelungen
  \item
    Einhaltung von Terminen
  \item
    Qualität der Beratung
  \item
    Umgang mit Reklamationen an
  \end{itemize}
\item
  \textbf{Ruf des Autohauses} (Reputationsqualität)

  \begin{itemize}
  \item
    Guter Ruf, Kompetenz
  \end{itemize}
\item
  \textbf{Persönliche Beziehungsqualität}

  \begin{itemize}
  \item
    Mitarbeiter - Kunde
  \end{itemize}
\item
  \textbf{Preiswahrnehmung}

  \begin{itemize}
  \item
    Gutes Preis-Leistungs-Verhältnis, Angebote, Transparenz
  \end{itemize}
\item
  \textbf{Kundenbindung}

  \begin{itemize}
  \item
    Ziel: langfristige Bindung
  \end{itemize}
\end{enumerate}

\textbf{Servicekonzepte, um die Kundenbindung zu verbessern}

\begin{itemize}
\item
  Werbung
\item
  Garantie und Kulanz
\item
  Hol- und Bring-Service
\item
  Reparatur-Finanzierung
\item
  Dienstleistungsangebote: Verkauf, Wartung
\end{itemize}

Bestandskunden halten vs.~Neukunden bewerben kostet 5 -- 6x mehr

\newpage

\textbf{Kundenarten}

\begin{enumerate}
\item
  \textbf{Laufkunde} (Kommt zufällig und hat keine Bindung)

  \begin{itemize}
  \item
    \emph{Bedeutung} Gering
  \item
    \emph{Erwartung des Kunden} Schnelle und zuverlässige Ausführung der
    Arbeit
  \item
    \emph{Maßnahmen} Keine
  \end{itemize}
\item
  \textbf{Dauerkunde} (nimmt gelegentlich Service in Anspruch)

  \begin{itemize}
  \item
    \emph{Bedeutung} Mittel
  \item
    \emph{Erwartung des Kunden} zuverlässig und preisgünstig
  \item
    \emph{Maßnahmen} Angebote an Kunden
  \end{itemize}
\item
  \textbf{Stammkunde} (lässt alle Arbeiten in der Werkstatt ausführen)

  \begin{itemize}
  \item
    \emph{Bedeutung} Hoch, Wachstum und Gewinn kann erwartet werden,
    Weiterempfehlung des Betriebs
  \item
    \emph{Erwartung des Kunden} persönliche Betreuung
  \item
    \emph{Maßnahmen} persönliche Ansprache
  \end{itemize}
\item
  \textbf{Großkunde} (Gesamten Fuhrpark warten)

  \begin{itemize}
  \item
    \emph{Bedeutung} sehr hoch
  \item
    \emph{Erwartung des Kunden} Schnelle und gute Ausführung, Kulanz
  \item
    \emph{Maßnahmen} Rabatt, Terminvereinbarung
  \end{itemize}
\end{enumerate}

\emph{Vorsicht bei Zahlungszielen} von 30 oder 60 Tage. \emph{Beispiel:}
Aldi legt bei einer Bank stundenweise / 28 Tage lang Geld an und lässt
das Geld für sich arbeiten.

\textbf{Beratungsgespräch} $\to$ \emph{Ziel:} Kundenwünsche ermitteln,
Kundenbindung und -gewinnung

\newpage

\section{Marketing}\label{marketing}

$\to$ \emph{Ziel:} verbesserte Qualität, Erhöhen der Marktanteile,
Gewinnen neuer Kunden, Verbesserung des Images

\subsection{Marktforschung}\label{marktforschung}

\begin{itemize}
\item
  Marktbeobachtung (Regelmäßige Untersuchungen auf Preise, Qualität und
  Quantität)
\item
  Marktanalyse (Einmalige Auswertung wichtiger Marktdaten)
\item
  Marktprognose (Aussage über voraussichtliche Marktentwicklung)
\end{itemize}

\textbf{Marktinformationen}

\begin{itemize}
\item
  Allgemeine Marktinformationen (Trends, Mode, Marktentwicklung,
  technischer Fortschritt)
\item
  Konkurrenzinformation (Dichte, Schwächen und Stärken, Ziele, Angebote)
\item
  Lieferanteninformationen (Dichte, Leistungen, Konditionen, Ansprüche)
\item
  Kundeninformationen (Kundenzahl, Kaufkraft und Einkommen,
  Kundenwünsche, Lebensstil, Produktkenntnisse)
\end{itemize}

\subsection{Marketing-Mix}\label{marketing-mix}

Marketing erfordert je nach Produkt andere Maßnamenskombinationen

\begin{enumerate}
\item
  Produkt- und Sortimentspolitik (Kundendienst, Sortimentsgestaltung,
  Produktveränderung)

  \begin{itemize}
  \item
    \textbf{Produktelemente}

    \begin{itemize}
    \item
      Kernprodukt (Kernvorteile)
    \item
      Formales Produkt (Markenname, Qualität, Produkteigenschaften,
      Styling, Verpackung)
    \item
      Erweitertes Produkt (Kostenlose Lieferung, Garantieleistung,
      Installation, Service)
    \end{itemize}
  \item
    \textbf{Produktlebenszyklus} Phasen

    \begin{itemize}
    \item
      Entwicklung (Entwicklungskosten)
    \item
      Einführungsphase (hoher Verlust)
    \item
      Wachstumsphase (Verbesserung)
    \item
      \textbf{Reifephase (hohe Gewinne)}
    \item
      Sättigungsphase (Gewinnrückgang)
    \item
      Rückgangsphase (geringe Gewinne)
    \end{itemize}
  \end{itemize}
\item
  Kommunikationspolitik (Werbemaßnahmen, Öffentlichkeitsarbeit,
  Verkaufsförderung)
\item
  Preis- und Konditionspolitik (marktbezogene Preisgestaltung, Liefer-
  und Zahlungsbedingungen)

  \begin{itemize}
  \item
    \textbf{Marktarten und Preisgestaltung}

    \begin{itemize}
    \item
      Angebotsmonopol \emph{Beispiel:} Bahn, früher: Deutsche Post,
      Telekom
    \item
      Nachfragemonopol \emph{Beispiel:} Rüstungsindustrie, Kampfpanzer
    \item
      Angebotsoligopol \emph{Beispiel:} Mobilfunkanbieter,
      Preisabsprachen
    \item
      Nachfrageoligopol \emph{Beispiel:}
    \item
      Polypol \emph{Beispiel:}
    \end{itemize}
  \end{itemize}
\item
  Distributionspolitik (Vertriebswege, Messen, Filialen, Vertreter)
\end{enumerate}

\newpage

\section{Recht}\label{recht}

\textbf{Welche Möglichkeit hat der Kunde, wenn er einen Mangel an seinem
neuen Fahrzeug feststellt?}

Käufer hat Recht

\begin{itemize}
\item
  auf Nacherfüllung (Reparatur oder Neulieferung)
\item
  Rücktritt
\item
  Minderung des Kaufpreises
\item
  Anspruch auf Schadenersatz statt der Leistung
\item
  Ersatz vergeblicher Leistungen
\end{itemize}

\textbf{Unterschied zwischen Garantie und Sachmängelhaftung}

Garantie ist eine freiwillige Leistung des Betriebes. Die
Garantielaufzeit kann frei mit dem Kunden vereinbart werden.

Die Sachmängelhaftung ist vom Gesetzgeber vorgeschrieben und ist 24
Monate beziehungsweise mit Einschränkung 12 Monate gültig (gebrauchte
Ware).

\textbf{Beweislast im Rahmen der Sachmängelhaftung}

\begin{itemize}
\item
  bis 6 Monate: Beweislast beim Unternehmen
\item
  nach 6 Monate: Unternehmen kann Beweislast auf den Kunden umkehren
\end{itemize}
