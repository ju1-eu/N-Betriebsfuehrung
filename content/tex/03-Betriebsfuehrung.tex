%ju 06-Jun-22 03-Betriebsfuehrung.tex
\section{Betriebsorganisation}\label{betriebsorganisation}

Ein Unternehmen ist auf die Optimierung des Gewinns ausgerichtet. Dies
wird erreicht durch den optimalen Einsatz von Mitarbeitern, Maschinen,
Material und Zeit.

\newpage

\subsection{Aufbauorganisation -- Geschäftsbereiche eines
Autohauses}\label{aufbauorganisation-geschaeftsbereiche-eines-autohauses}

\textbf{Organigramm} $\to$ Hierarchisch strukturiert,
Organisationsstruktur, Weisungsbeziehungen

\textbf{Softskills} $\to$ Selbstsicherheit, Selbstständigkeit,
Entscheidungsfähigkeit

\begin{enumerate}
\item
  \textbf{Geschäftsleitung}

  \begin{itemize}
  \item
    \emph{Aufgaben} Kundenbeschwerde über eine zu hohe Rechnung,
    Betriebsführung, Planung und Organisation
  \item
    \emph{Funktionen} bestimmt Geschäftspolitik und legt die Zielsetzung
    des Autohauses fest
  \end{itemize}
\item
  \textbf{Kundendienst}

  \begin{itemize}
  \item
    \emph{Aufgaben} Annahme von Reparaturen, technische Beratung des
    Kunden, Fahrzeugübergabe an Kunden, Abwicklung von Garantiefällen
  \item
    \emph{Funktionen} Schnittstelle zwischen Kunden und Werkstatt
  \end{itemize}
\item
  \textbf{Kfz-Werkstatt}

  \begin{itemize}
  \item
    \emph{Aufgaben} Durchführung von Reparaturen und Wartungsarbeiten,
    Einbau von Zubehör
  \item
    \emph{Funktionen} Durchführung der Werkstattarbeiten
  \end{itemize}
\item
  \textbf{Teiledienst}

  \begin{itemize}
  \item
    \emph{Aufgaben} Verwaltung von den Ersatzteilen und Zubehör, Ausgabe
    von Teilen, Verkauf von Teilen
  \item
    \emph{Funktionen} Verwaltung eines Ersatzteile- und
    Zubehörsortiments
  \end{itemize}
\item
  \textbf{Verkauf}

  \begin{itemize}
  \item
    \emph{Aufgaben} Kundenberatung, Neuwagenverkauf, Verkauf von
    Gebrauchtwagen, Fahrzeugauslieferung und -übergabe, Bewertung von
    Gebrauchtwagen
  \item
    \emph{Funktionen} Umsatz von Fahrzeugen
  \end{itemize}
\item
  \textbf{Verwaltung}

  \begin{itemize}
  \item
    \emph{Aufgaben} Zahlungserinnerung einer nicht gezahlten Rechnung an
    den Kunden, Buchhaltung, Abwicklung von Geschäften mit Lieferanten
    und Herstellern, Lohn- und Gehaltsabrechnung
  \item
    \emph{Funktionen} kaufmännische Aufgaben
  \end{itemize}
\end{enumerate}

\newpage

\subsection{Kunden und Betrieb}\label{kunden-und-betrieb}

\textbf{Kundenorientierung} ist die Ausrichtung des Denkens und Handelns
der Mitarbeiter auf den Kunden und seine Bedürfnisse. Macht das
wirtschaftlich Sinn? Kundenanforderungen zu erfüllen oder Erwartungen
des Kunden zu übertreffen.

\textbf{Was beeinflusst die Kundenzufriedenheit? Nenne Merkmale}

\begin{enumerate}
\item
  \textbf{Technische Produktqualität}

  \begin{itemize}
  \item
    Verarbeitung und Reparaturanfälligkeit
  \item
    Ausführung von Wartungs- und Reparaturarbeiten
  \end{itemize}
\item
  \textbf{Servicequalität}

  \begin{itemize}
  \item
    Kulanzregelungen
  \item
    Einhaltung von Terminen
  \item
    Qualität der Beratung
  \item
    Umgang mit Reklamationen an
  \end{itemize}
\item
  \textbf{Ruf des Autohauses} (Reputationsqualität)

  \begin{itemize}
  \item
    Guter Ruf, Kompetenz
  \end{itemize}
\item
  \textbf{Persönliche Beziehungsqualität}

  \begin{itemize}
  \item
    Mitarbeiter - Kunde
  \end{itemize}
\item
  \textbf{Preiswahrnehmung}

  \begin{itemize}
  \item
    Gutes Preis-Leistungs-Verhältnis, Angebote, Transparenz
  \end{itemize}
\item
  \textbf{Kundenbindung}

  \begin{itemize}
  \item
    Ziel: langfristige Bindung
  \end{itemize}
\end{enumerate}

\textbf{Servicekonzepte, um die Kundenbindung zu verbessern}

\begin{itemize}
\item
  Werbung
\item
  Garantie und Kulanz
\item
  Hol- und Bring-Service
\item
  Reparatur-Finanzierung
\item
  Dienstleistungsangebote: Verkauf, Wartung
\end{itemize}

Bestandskunden halten vs.~Neukunden bewerben kostet 5 -- 6x mehr

\newpage

\textbf{Kundenarten}

\begin{enumerate}
\item
  \textbf{Laufkunde} (Kommt zufällig und hat keine Bindung)

  \begin{itemize}
  \item
    \emph{Bedeutung} Gering
  \item
    \emph{Erwartung des Kunden} Schnelle und zuverlässige Ausführung der
    Arbeit
  \item
    \emph{Maßnahmen} Keine
  \end{itemize}
\item
  \textbf{Dauerkunde} (nimmt gelegentlich Service in Anspruch)

  \begin{itemize}
  \item
    \emph{Bedeutung} Mittel
  \item
    \emph{Erwartung des Kunden} zuverlässig und preisgünstig
  \item
    \emph{Maßnahmen} Angebote an Kunden
  \end{itemize}
\item
  \textbf{Stammkunde} (lässt alle Arbeiten in der Werkstatt ausführen)

  \begin{itemize}
  \item
    \emph{Bedeutung} Hoch, Wachstum und Gewinn kann erwartet werden,
    Weiterempfehlung des Betriebs
  \item
    \emph{Erwartung des Kunden} persönliche Betreuung
  \item
    \emph{Maßnahmen} persönliche Ansprache
  \end{itemize}
\item
  \textbf{Großkunde} (Gesamten Fuhrpark warten)

  \begin{itemize}
  \item
    \emph{Bedeutung} sehr hoch
  \item
    \emph{Erwartung des Kunden} Schnelle und gute Ausführung, Kulanz
  \item
    \emph{Maßnahmen} Rabatt, Terminvereinbarung
  \end{itemize}
\end{enumerate}

\emph{Vorsicht bei Zahlungszielen} von 30 oder 60 Tage. \emph{Beispiel:}
Aldi legt bei einer Bank stundenweise / 28 Tage lang Geld an und lässt
das Geld für sich arbeiten.

\textbf{Beratungsgespräch} $\to$ \emph{Ziel:} Kundenwünsche ermitteln,
Kundenbindung und -gewinnung

\newpage

\section{Marketing}\label{marketing}

Vgl. Marketing S. 142 \textcite{heiser:2017:betriebsfuhrung}.

$\to$ \emph{Ziel:} verbesserte Qualität, Erhöhen der Marktanteile,
Gewinnen neuer Kunden, Verbesserung des Images

\subsection{Marktforschung}\label{marktforschung}

\begin{enumerate}
\item
  Marktbeobachtung (Regelmäßige Untersuchungen auf Preise, Qualität und
  Quantität)
\item
  Marktanalyse (Einmalige Auswertung wichtiger Marktdaten)
\item
  Marktprognose (Aussage über voraussichtliche Marktentwicklung)
\end{enumerate}

\textbf{Marktinformationen}

\begin{enumerate}
\item
  \textbf{Allgemeine Marktinformationen} (Trends, Mode,
  Marktentwicklung, technischer Fortschritt, wirtschaftliche Entwicklung
  und Lage)
\item
  \textbf{Konkurrenzinformation} (Dichte, Schwächen und Stärken, Ziele,
  Angebote)
\item
  \textbf{Lieferanteninformationen} (Dichte, Leistungen, Konditionen,
  Ansprüche)
\item
  \textbf{Kundeninformationen} (Kundenzahl, Kaufkraft und Einkommen,
  Kundenwünsche, Lebensstil, Produktkenntnisse)
\end{enumerate}

\subsection{Marketing-Mix}\label{marketing-mix}

Dieser bezeichnet die Koordination verschiedener Marketing Aktivitäten,
um die Marketingstrategien eines Unternehmens umzusetzen und die Kunden
gezielt anzusprechen. Die klassische Theorie unterscheidet zwischen vier
verschiedenen Instrumenten (den 4Ps).

Ein Unternehmen entwirft also eine Strategie, welches Produkt und zu
welchem Preis dem Kunden angeboten wird, über welche Absatzwege der
Verkauf stattfindet und wie man auf das Gut aufmerksam macht.

\begin{enumerate}
\item
  \textbf{Produktpolitik} (Kundendienst, Sortimentsgestaltung,
  Produktveränderung)

  \begin{itemize}
  \item
    Das Produkt sollte so gestaltet werden, dass es den Bedürfnissen des
    Kunden gerecht wird.
  \item
    \textbf{Produktelemente}

    \begin{itemize}
    \item
      Kernprodukt (Kernvorteile)
    \item
      Formales Produkt (Markenname, Qualität, Produkteigenschaften,
      Styling, Verpackung)
    \item
      Erweitertes Produkt (Kostenlose Lieferung, Garantieleistung,
      Installation, Service)
    \end{itemize}
  \item
    \textbf{Produktlebenszyklus} (Phasen)

    \begin{itemize}
    \item
      Entwicklung (Entwicklungskosten)
    \item
      Einführungsphase (hoher Verlust)
    \item
      Wachstumsphase (Verbesserung)
    \item
      \textbf{Reifephase (hohe Gewinne)}
    \item
      Sättigungsphase (Gewinnrückgang)
    \item
      Rückgangphase (geringe Gewinne)
    \end{itemize}
  \item
    Beispiel:

    \begin{itemize}
    \item
      Welches Produkt biete ich meiner Zielgruppe an?
    \item
      Welches Produkt kann ich aus meinem Sortiment entfernen?
    \item
      Welche Eigenschaften soll mein Produkt vorweisen können (Design,
      Qualität, Verpackung)?
    \end{itemize}
  \end{itemize}
\item
  \textbf{Kommunikationspolitik} (Werbemaßnahmen, Öffentlichkeitsarbeit,
  Verkaufsförderung)

  \begin{itemize}
  \item
    Wie soll Produkt am besten präsentiert werden, Beispiel: durch
    klassische Werbung oder Social Media Marketing.
  \item
    Wenn sich das eigene Produkt von der Konkurrenz abgrenzt und
    heraussticht, bleibt es dem Endverbraucher eher im Gedächtnis.
  \item
    Ziel: Vertrauen des Kunden gewinnen und ihn langfristig an das
    Unternehmen binden.
  \item
    vgl. \textbf{AIDA} erklärt die Kaufentscheidung
  \item
    \textbf{Corporate Identity} Unternehmensphilosophie (Wir-Gefühl)
  \item
    \textbf{Corporate Design} einheitliches Erscheinungsbild (Beispiel:
    Gestaltung des Logos, Hausfarbe, Schriftart, Berufskleidung, Briefe)
  \item
    Beispiel:

    \begin{itemize}
    \item
      Welchen Kommunikationsweg wähle ich?
    \item
      Betreibe ich klassische Werbung via TV-Spots, Radio oder
      Printmedien?
    \item
      Möchte ich auf Social-Media-Kanälen präsent sein?
    \item
      Mache ich von Direct-Marketing (z. B. Kunden gezielt anschreiben)
      Gebrauch?
    \item
      Spreche ich meine Kunden durch Sponsoring an?
    \item
      Präsentiere ich mein Produkt auf einer Messe?
    \end{itemize}
  \end{itemize}
\item
  \textbf{Preispolitik} (marktbezogene Preisgestaltung, Liefer- und
  Zahlungsbedingungen)

  \begin{itemize}
  \item
    Bei der Gestaltung des Preises müssen dabei unterschiedliche Aspekte
    wie anfallende Kosten, Nachfrage der Zielgruppen und Konkurrenz
    berücksichtigt werden. Der Verkaufspreis muss von den Kunden
    akzeptiert werden, aber dennoch wettbewerbsfähig bleiben. Das Ziel
    ist es natürlich, den Gewinn zu maximieren.
  \item
    \textbf{Marktarten und Preisbildung} hängt von der Marktsituation ab

    \begin{itemize}
    \item
      Angebotsmonopol \emph{Beispiel:} Bahn, früher: Deutsche Post,
      Telekom
    \item
      Nachfragemonopol \emph{Beispiel:} Rüstungsindustrie, Kampfpanzer
    \item
      Angebotsoligopol \emph{Beispiel:} Mobilfunkanbieter,
      Preisabsprachen
    \item
      Nachfrageoligopol
    \item
      Polypol
    \end{itemize}
  \item
    Beispiel:

    \begin{itemize}
    \item
      Welchen Preis verlange ich für mein Produkt?
    \item
      Biete ich Rabatte an?
    \item
      Für welche Zahlungskonditionen entscheide ich mich?
    \end{itemize}
  \end{itemize}
\item
  \textbf{Distributionspolitik} (Absatzwege, Messen, Filialen,
  Vertreter)

  \begin{itemize}
  \item
    wie das Produkt am besten zum Endverbraucher gelangt.
  \item
    Beispiel:

    \begin{itemize}
    \item
      Welchen Vertriebsweg (direkt oder indirekt) wähle ich?
    \item
      Welche Vertriebskanäle (z.B. eigenes Geschäft, Internet, vgl.
      \textbf{Franchising}) verwende ich?
    \item
      Kooperiere ich mit einem Vertriebspartner (z.B. Großhändler) und
      gebe den Vertrieb an ihn ab?
    \end{itemize}
  \end{itemize}
\end{enumerate}

\textbf{Was ist Franchising?}

\begin{itemize}
\item
  Franchising ist ein auf Partnerschaft basierendes Vertriebssystem
  zwischen einem bestehenden Unternehmen, dem sogenannten
  \emph{Franchisegeber}, und einem Neuunternehmer, dem sogenannten
  \emph{Franchisenehmer}.
\item
  Der \emph{Franchisenehmer} zahlt eine einmalige oder fortlaufende
  Gebühr an den Franchisegeber.
\item
  Als Gegenleistung erlangt der \emph{Franchisenehmer} das Recht, Name,
  Design und Geschäftsidee des anderen Unternehmens für einen bestimmten
  Zeitraum nutzen zu dürfen
\item
  Die Franchisegebühren fallen also für Lizenzen und Nutzungsrechte an
  und binden den Franchisenehmer an den -geber.
\end{itemize}

\textbf{Franchisenehmer}

\textbf{Vorteile}

\begin{enumerate}
\item
  Beginn der Selbstständigkeit ein vermindertes Risiko, da dir ein
  erfahrenes Unternehmen zur Seite steht. Sein Wissen gibt der
  Franchisegeber schließlich immer direkt an seine Franchisenehmer ab.
\item
  Bekanntheit des Franchisegebers, was dir ein positives Image
  verschafft.
\item
  Marketingplan nutzen und ein ausgefeiltes Unternehmenskonzept, was
  bereits erfolgreich funktioniert.
\item
  direkt mit einer höheren Kreditwürdigkeit gegenüber Banken starten.
\end{enumerate}

\textbf{Nachteile}

\begin{enumerate}
\item
  wenig Raum für eigene Kreativität und Möglichkeiten zur Mitgestaltung
  gibt.
\item
  zum Teil hohe Prozentsätze an den Franchisegeber, also den Urheber
  gehen.
\end{enumerate}

\textbf{Franchisegebers}

\textbf{Vorteile}

\begin{enumerate}
\item
  Durch die Zusammenarbeit sein Bekanntheitsgrad gesteigert wird und ein
  einheitlicher Markenauftritt möglich wird.
\item
  profitiert von den monatlichen Einnahmen, welche er von dir bekommt
\item
  bessere Fokussierung auf Arbeitsbereiche möglich, da sich der
  Franchisegeber nicht um alle Zweigstellen allein kümmern muss. Dies
  fördert gleichzeitig die Marktdeckung.
\end{enumerate}

\textbf{Nachteile}

\begin{enumerate}
\item
  Durch die Arbeitsteilung verliert der Franchisegeber den direkten
  Kundenkontakt außerhalb seines Zuständigkeitsbereichs.
\item
  hoher Kontrollaufwand nötig ist, um Einheitlichkeit und Identität des
  Konzepts sicherzustellen.
\end{enumerate}

\textbf{AIDA} Modell zeigt die vier Stufen, die ein Konsument
durchläuft, bevor er sich für den Kauf eines Produkts entscheidet.

\begin{enumerate}
\item
  \textbf{A Attention} -- Aufmerksamkeit erzeugen

  \begin{itemize}
  \item
    Werbung hat die Aufgabe, die Aufmerksamkeit der gewünschten
    Zielgruppe zu gewinnen.
  \item
    durch auffällige Farben, einprägsame Werbesprüche oder Sonderrabatte
  \item
    Beispiel: Nachhaltige Sneaker um 70 \% reduziert.
  \end{itemize}
\item
  \textbf{I Interest} -- Interesse wecken

  \begin{itemize}
  \item
    So kann sich das Produkt langfristig im Gedächtnis deiner Kunden
    verankern.
  \item
    Produktbroschüren, Flyer oder Videoclips, die Detailinformationen
    liefern
  \item
    Beispiel: verschiedene Arten von nachhaltigen Sneakern in
    unterschiedlichen Farben, Größen und Modellen zu 70 \% reduziert
    sind
  \end{itemize}
\item
  \textbf{D Desire} -- Verlangen, Wunsch auslösen

  \begin{itemize}
  \item
    durch Marketing und emotionale oder rationale Werbebotschaften
    erreichen.
  \item
    Beispiel: Deine Sneaker eignen sich für Städtereisen als auch für
    sportliche Aktivitäten. Gleichzeitig sind sie langlebig, sehen gut
    aus und helfen der Umwelt. Einen besseren Freizeitschuh kann man für
    den Preis nirgendwo finden!
  \end{itemize}
\item
  \textbf{A Action} -- Handlung, Kauf

  \begin{itemize}
  \item
    mit der sogenannten Call-to-Action (Handlungsaufforderung).
  \item
    durch einen Kauf-Button am Ende einer Landingpage im Internet oder
    den Verweis zur Bestellhotline deines Produkts erreichen.
  \item
    Beispiel: Button mit Aufforderung: Jetzt direkt zuschlagen!
  \end{itemize}
\end{enumerate}

\newpage

\section{Recht}\label{recht}

\textbf{Welche Möglichkeit hat der Kunde, wenn er einen Mangel an seinem
neuen Fahrzeug feststellt?}

Käufer hat Recht

\begin{itemize}
\item
  auf Nacherfüllung (Reparatur oder Neulieferung)
\item
  Rücktritt
\item
  Minderung des Kaufpreises
\item
  Anspruch auf Schadenersatz statt der Leistung
\item
  Ersatz vergeblicher Leistungen
\end{itemize}

\textbf{Unterschied zwischen Garantie und Sachmängelhaftung}

Garantie ist eine freiwillige Leistung des Betriebs. Die
Garantielaufzeit kann frei mit dem Kunden vereinbart werden.

Die Sachmängelhaftung ist vom Gesetzgeber vorgeschrieben und ist 24
Monate beziehungsweise mit Einschränkung 12 Monate gültig (gebrauchte
Ware).

\textbf{Beweislast im Rahmen der Sachmängelhaftung}

\begin{itemize}
\item
  bis 6 Monate: Beweislast beim Unternehmen
\item
  nach 6 Monate: Unternehmen kann Beweislast auf den Kunden umkehren
\end{itemize}
