%ju 27-Mär-22 U01-Ersatzteilpreiskalkulation-KI-HSP-KF-Loesung.tex
\textbf{Ü1 - Ersatzteilpreiskalkulation - KI - HSP - KF}

\textbf{Aufgabe 1)}

\begin{enumerate}
\def\labelenumi{\alph{enumi})}
\item
  \textbf{Gemeinkostenzuschlag}
\end{enumerate}

\begin{itemize}
\item
  GK = Fl x GKZs / 100 \%
\item
  GK = 13,50 x 350 \% / 100 \% = 47,25 €/h
\end{itemize}

\begin{enumerate}
\def\labelenumi{\alph{enumi})}
\setcounter{enumi}{1}
\item
  \textbf{Selbstkostenanteil}
\end{enumerate}

\begin{itemize}
\item
  Seko = Fl + GK
\item
  Seko = 13,50 €/h + 47,25 €/h = 60,75 €/h
\end{itemize}

\begin{enumerate}
\def\labelenumi{\alph{enumi})}
\setcounter{enumi}{2}
\item
  \textbf{Gewinnzuschlag} (in €)
\end{enumerate}

\begin{itemize}
\item
  Gewinn = Seko x GWZs / 100 \%
\item
  Gewinn = 60,75 €/h x 9 \% / 100 \% = 5,47 €/h
\end{itemize}

\begin{enumerate}
\def\labelenumi{\alph{enumi})}
\setcounter{enumi}{3}
\item
  \textbf{Stundenverrechnungssatz}
\end{enumerate}

\begin{itemize}
\item
  St-Vs = Seko + Gewinn
\item
  St-Vs = 60,75 €/h + 5,47 €/h = 66,22 €/h
\end{itemize}

\begin{enumerate}
\def\labelenumi{\alph{enumi})}
\setcounter{enumi}{4}
\item
  \textbf{Kostenindex}
\end{enumerate}

\begin{itemize}
\item
  KI = St-Vs / WSL
\item
  KI = 66,22 €/h / 13,50 €/h = 4,91 €/h
\end{itemize}

\textbf{Aufgabe 2)}

\begin{enumerate}
\def\labelenumi{\alph{enumi})}
\item
  \textbf{Selbstkosten}
\end{enumerate}

\begin{itemize}
\item
  UE = Fl (produktiv) x KI
\item
  UE = 25.200 € x 4,25 = 107.100 €
\item
  Seko = UE - Gewinn
\item
  Seko = 107.100 € - 8.500 € = 98.600 €
\end{itemize}

\begin{enumerate}
\def\labelenumi{\alph{enumi})}
\setcounter{enumi}{1}
\item
  \textbf{Fertigungsgemeinkosten}
\end{enumerate}

\begin{itemize}
\item
  GK = Seko - Fl
\item
  GK = 98.600 - 25.200 = 73.400 €
\end{itemize}

\begin{enumerate}
\def\labelenumi{\alph{enumi})}
\setcounter{enumi}{2}
\item
  \textbf{Gewinnzuschlagsatz} (in \%)
\end{enumerate}

\begin{itemize}
\item
  GWZs = GW x 100 \% / Seko
\item
  GWZs = 8.500 € x 100 \% / 98.600 € = 8,62 \%
\end{itemize}

\textbf{Aufgabe 3)}

Vgl. Übungsaufgaben / Excel
>>U01-Ersatzteilpreiskalkulation-A3+4-Loesung.pdf<<

\begin{enumerate}
\def\labelenumi{\alph{enumi})}
\item
  \textbf{Zielverkaufspreis}
\end{enumerate}

\begin{itemize}
\item
  ZVP = BVP + KSk
\item
  ZVP = 1.550 € x 100 \% / 98 \% = 1.581,63 €

  \begin{itemize}
  \item
    NR) 100 \% - 2 \% = 98 \%
  \end{itemize}
\end{itemize}

\begin{enumerate}
\def\labelenumi{\alph{enumi})}
\setcounter{enumi}{1}
\item
  \textbf{Listenverkaufspreis}
\end{enumerate}

\begin{itemize}
\item
  LVP = ZVP + KRa
\item
  LVP = 1.581 € x 100 \% / 88 = 1.797,31 €

  \begin{itemize}
  \item
    NR) 100 \% - 12 \% = 88 \%
  \end{itemize}
\end{itemize}

\begin{enumerate}
\def\labelenumi{\alph{enumi})}
\setcounter{enumi}{2}
\item
  \textbf{Rechnungsbetrag ohne Rabatt}
\end{enumerate}

\begin{itemize}
\item
  = LVP + USt
\item
  = 1.797,31 € + 341,49 € (19 \%) = 2.138,80 €
\end{itemize}

\begin{enumerate}
\def\labelenumi{\alph{enumi})}
\setcounter{enumi}{3}
\item
  \textbf{Kalkulationsfaktor}
\end{enumerate}

\begin{itemize}
\item
  KF = LVP / BP
\item
  KF = 1.797,31 € / 975,00 € = 1,84
\end{itemize}

\begin{enumerate}
\def\labelenumi{\alph{enumi})}
\setcounter{enumi}{4}
\item
  \textbf{Handelsspanne} (in €)
\end{enumerate}

\begin{itemize}
\item
  HSP = LVP - BP
\item
  HSP = 1.797,31 € - 975,00 € = 822,31 €
\end{itemize}

\begin{enumerate}
\def\labelenumi{\alph{enumi})}
\setcounter{enumi}{5}
\item
  \textbf{Handelsspanne} (in \%)
\end{enumerate}

\begin{itemize}
\item
  HSP = HSP x 100 \% / LVP
\item
  HSP = 822,31 € x 100 \% / 1.797,31 € = 45,75 \%

  \begin{itemize}
  \item
    LVP (100 \%) - HSP (45,75 \%) = BP (54,25 \%)
  \item
    Schnell rechnen, Überschlagswert:

    \begin{itemize}
    \item
      100 € (Betrag) x 1,84 (KF) = 184 € x 0,88 (Rabatt) = 161,92 €
      (Kunde)
    \end{itemize}
  \end{itemize}
\end{itemize}

\textbf{Aufgabe 4)}

Vgl. Übungsaufgaben / Excel
>>U01-Ersatzteilpreiskalkulation-A3+4-Loesung.pdf<<

\begin{enumerate}
\def\labelenumi{\alph{enumi})}
\item
  \textbf{Listenverkaufspreis}
\end{enumerate}

\lstset{language=Python}% C, TeX, Bash, Python 
\begin{lstlisting}[
	%caption={}, label={code:}%% anpassen
]
  BP                        35,00 EUR
+ GK       45 %             15,75 EUR
= Seko                      50,75 EUR
+ Gewinn    6 % (auf 100)    3,05 EUR
= BVP                       53,80 EUR // 98 %
+ Skonto    2 % (in 100)    
= ZVP                       54,90 EUR // 90 %
+ Rabatt   10 % (in 100)   
--------------- 
= LVP                       61,00 EUR
\end{lstlisting}

\begin{enumerate}
\def\labelenumi{\alph{enumi})}
\setcounter{enumi}{1}
\item
  \textbf{Handelsspanne}
\end{enumerate}

\begin{itemize}
\item
  HSP (in €) = LVP - BP
\item
  HSP (in €) = 61,00 € - 35 € = 26 €
\item
  HSP (in \%) = HSP x 100 \% / LVP
\item
  HSP (in \%) = 26 € x 100 \% / 61 € = 42,62 \%
\end{itemize}

\begin{enumerate}
\def\labelenumi{\alph{enumi})}
\setcounter{enumi}{2}
\item
  \textbf{Kalkulationsfaktor}
\end{enumerate}

\begin{itemize}
\item
  KF = LVP / BP
\item
  KF = 61 € / 35 € = 1,74
\end{itemize}
