%ju 28-Mai-22 U03-Kundenrechnung-Kostenvoranschlag-AW-Vs-UR-WI-Loesung.tex
\textbf{Ü3 - Kundenrechnung - Kostenvoranschlag - AW-VS - UR - WI}

\textbf{Aufgabe 1)}

Kundenrechnung Vgl. Übungsaufgaben / Excel
>>U03-Kundenrechnung-A1-Loesung.pdf<<

\textbf{Aufgabe 2)}

\begin{enumerate}
\def\labelenumi{\alph{enumi})}
\item
  AW-Verrechnungssatz und Werkstattindex
\end{enumerate}

\begin{itemize}
\item
  GK/h = WSL x FGKZs / 100 \%
\item
  GK/h = 14,75 €/h x 2,80 = 41,30 €/h
\item
  Seko/h = WSL + GK/h
\item
  Seko/h = 14,75 €/h + 41,30 €/h = 56,05 €/h
\item
  Gewinn/h (Werkstatt) = Seko/h x GWZs / 100 \%
\item
  Gewinn/h (Werkstatt) = 56,05 €/h x 0,08 = 4,48 €/h
\item
  StVs = Seko/h + Gewinn/h
\item
  StVs = 56,05 €/h + 4,48 €/h = 60,53 €/h
\end{itemize}

\textbf{AW-Verrechnungssatz} (in €/AW)

\begin{itemize}
\item
  AW-Vs = StVs / WF
\item
  AW-Vs = 60,53 €/h / 14 AW/h = 4,32 €/AW
\end{itemize}

\textbf{Werkstattindex}

\begin{itemize}
\item
  WI = StVs / WSL
\item
  WI = 60,53 €/h / 14,75 €/h = 4,32
\end{itemize}

\begin{enumerate}
\def\labelenumi{\alph{enumi})}
\setcounter{enumi}{1}
\item
  \textbf{Preis der Ersatzteile und AT-Teile}
\end{enumerate}

\lstset{language=Python}% C, TeX, Bash, Python 
\begin{lstlisting}[
	%caption={}, label={code:}%% anpassen
]
Bezugspreis Ersatzteile =  625,00 EUR
MGK             75 %    =  468,75 EUR
Seko                    = 1093,75 EUR
Gewinn (ET)     11 %    =  120,31 EUR
-------------
VK Preis ET             = 1214,06 EUR
\end{lstlisting}

\lstset{language=Python}% C, TeX, Bash, Python 
\begin{lstlisting}[
	%caption={}, label={code:}%% anpassen
]
Bezugspreis AT-Teile    =  125,00 EUR
MGK             75 %    =   93,75 EUR
Seko                    =  218,75 EUR
Gewinn (AT)     11 %    =   24,06 EUR
-------------
VK Preis AT             =  242,81 EUR
\end{lstlisting}

\begin{enumerate}
\def\labelenumi{\alph{enumi})}
\setcounter{enumi}{2}
\item
  \textbf{Preis der Fremdarbeit - Lackierung}
\end{enumerate}

\lstset{language=Python}% C, TeX, Bash, Python 
\begin{lstlisting}[
	%caption={}, label={code:}%% anpassen
]
Einstandspreis Lackierung =  575,00 EUR
Gewinn (Fremd)    12 %    =   69,00 EUR
-------------
Lackierung                =  644,00 EUR
\end{lstlisting}

\begin{enumerate}
\def\labelenumi{\alph{enumi})}
\setcounter{enumi}{3}
\item
  \textbf{Gesamtgewinn}
\end{enumerate}

\begin{itemize}
\item
  $Gewinn_{gesamt}$ = (115 AW x Gewinn (Werkstatt) / WF) + Gewinn (ET)
  + Gewinn (AT) + Gewinn (Fremd)
\item
  $Gewinn_{gesamt}$ = (115 AW x 4,48 €/h / 14 AW/h) + 120,31 € + 24,06
  € + 69 € = 250,17 €
\end{itemize}

\begin{enumerate}
\def\labelenumi{\alph{enumi})}
\setcounter{enumi}{4}
\item
  \textbf{Umsatzrendite}
\end{enumerate}

\begin{itemize}
\item
  UR = GW x 100 \% / Erlöse
\item
  UR = 250,17 € x 100 \% / 2.597,67 € = 9,63 \%
\end{itemize}

\begin{enumerate}
\def\labelenumi{\alph{enumi})}
\setcounter{enumi}{5}
\item
  \textbf{Kostenvoranschlag}
\end{enumerate}

\lstset{language=Python}% C, TeX, Bash, Python 
\begin{lstlisting}[
	%caption={}, label={code:}%% anpassen
]
Lohnarbeiten = 115 AW x 4,32 EUR/AW  =   496,80 EUR
Ersatzteile                          = 1.214,06 EUR
AT-Teile                             =   242,81 EUR
Lackierung                           =   644,00 EUR
Zwischensumme                        = 2.597,67 EUR
Mehrwertsteuer                 19 %  =   493,56 EUR
Mehrwertsteuer für Tauschteile 19 %  =     4,61 EUR
-------------------
Rechnungsbetrag                      = 3.095,84 EUR
\end{lstlisting}

\textbf{Aufgabe 3)}

\begin{enumerate}
\def\labelenumi{\alph{enumi})}
\item
  \textbf{Fertigungsgemeinkostenzuschlagsatz} (in \%)
\end{enumerate}

\begin{itemize}
\item
  FGKZs = FGK x 100 \% / FL
\item
  FGKZs = 197.250,00 x 100 \% / 81.500,00 € = 242,02 \%
\end{itemize}

\begin{enumerate}
\def\labelenumi{\alph{enumi})}
\setcounter{enumi}{1}
\item
  \textbf{Gewinnzuschlagsatz} (in \%)
\end{enumerate}

\begin{itemize}
\item
  GW = UE - Seko
\item
  GW = 311.250,00 € - 278.750,00 € = 32.500,00 €
\item
  Seko = FL + GK
\item
  Seko = 81.500,00 € + 197.250,00 € = 278.750,00 €
\item
  GWZs = GW x 100 \% / SEKO
\item
  GWZs = 32.500,00 € x 100 \% / 278.750,00 € = 11,66 \%
\end{itemize}

\begin{enumerate}
\def\labelenumi{\alph{enumi})}
\setcounter{enumi}{2}
\item
  \textbf{Umsatzrendite} (in \%)\\
\end{enumerate}

\begin{itemize}
\item
  UR = GW x 100 \% / UE
\item
  UR = 32.500,00 € x 100 \% / 311.250,00 € = 10,44 \%
\end{itemize}

\begin{enumerate}
\def\labelenumi{\alph{enumi})}
\setcounter{enumi}{3}
\item
  \textbf{Werkstattindex} WI = KI
\end{enumerate}

\begin{itemize}
\item
  KI = Umsatzerlöse / Fertigungslöhne
\item
  KI = 311.250,00 € / 81.500,00 € = 3,8190
\end{itemize}

\textbf{Aufgabe 4)}

LE = Lohnerlös

\begin{enumerate}
\def\labelenumi{\alph{enumi})}
\item
  \textbf{Kostenindex}
\end{enumerate}

\begin{itemize}
\item
  Soll-UE = StVs x LE (in h)
\item
  Soll-UE = 58,00 €/h x 6750 h = 391.500,00 €
\item
  $FL_{neu}$ = $FL_{bisher}$ + 1,75 \%
\item
  $FL_{neu}$ = 99.200,00 € x 1,0175 = 100.936,00 €
\item
  $GK_{neu}$ = $GK_{bisher}$ + 6 \%
\item
  $GK_{neu}$ = 208.000,00 € x 1,06 = 220.480,00 €
\item
  $Seko_{neu}$ = $FL_{neu}$ + GK\_neu
\item
  $Seko_{neu}$ = 100.936,00 + 220.480,00 € = 321.416,00 €
\item
  $GW_{neu}$ = Soll-UE - Seko\_neu
\item
  $GW_{neu}$ = 391.500,00 € - 321.416,00€ = 70.084,00 €
\item
  KI = Erlöse / Fertigungslöhne
\item
  KI = 391.500,00 € / 100.936,00 € = 3,879
\end{itemize}

\begin{enumerate}
\def\labelenumi{\alph{enumi})}
\setcounter{enumi}{1}
\item
  \textbf{Werkstattschnittlohn} (in €/h)
\end{enumerate}

\begin{itemize}
\item
  WSL = Fertigungslöhne / LE (in h)
\item
  WSL = 100.936,00 € / 6750 h = 14,95 €/h
\end{itemize}

\begin{enumerate}
\def\labelenumi{\alph{enumi})}
\setcounter{enumi}{2}
\item
  \textbf{Fertigungsgemeinkostenzuschlagsatz} (in \%)
\end{enumerate}

\begin{itemize}
\item
  GKZs = GW x 100 \% / FL
\item
  GKZs = 220.480,00 € x 100 \% / 100.936,00 € = 218,44 \%
\end{itemize}

\begin{enumerate}
\def\labelenumi{\alph{enumi})}
\setcounter{enumi}{3}
\item
  \textbf{Gewinnzuschlagsatz} (in \%)
\end{enumerate}

\begin{itemize}
\item
  GWZs = GW x 100 \% / SEKO
\item
  GWZs = 70.084,00 € x 100 \% / 321.416,00 € = 21,80 \%
\end{itemize}

\begin{enumerate}
\def\labelenumi{\alph{enumi})}
\setcounter{enumi}{4}
\item
  \textbf{Umsatzrendite} (in \%)
\end{enumerate}

\begin{itemize}
\item
  UR = GW x 100 \% / UE
\item
  UR = 70.084,00 € x 100 \% / 391.500,00 € = 17,90 \%
\end{itemize}
