%ju 28-Mai-22 README.tex
\section{Readme}\label{readme}

Erstellt Webseiten \& Latex-Files mit Markdown und Pandoc. Projekt wurde
getestet unter >>iMac<<

\subsection{Kurzbefehle}\label{kurzbefehle}

\textbf{Terminal} öffnen

\lstset{language=Python}% C, TeX, Bash, Python 
\begin{lstlisting}[
	%caption={}, label={code:}%% anpassen
]
# Schreiben in Markdown, Illustrator für Vektorgrafiken und Excel für Tabellen

./projekt.sh  # Schritt 2, 3, 5
##########################################################
    0) Projekt aufräumen
    1) Projekt erstellen
    2) Markdown in (tex, html5) + sed (Suchen/Ersetzen)
    3) Kapitel erstellen + Scripte ausführen
    4) Fotos optimieren (Web, Latex)
    5) www + index.html
    6) git init
    7) git status + git log
    8) Git-Version erstellen
    9) Backup + Archiv erstellen
    10) Beenden?
##########################################################

# PDF erstellen
make distclean
make
make clean

# Git Version
git add .
git commit -a
git push

# Backup
./projekt.sh  # Schritt 9
\end{lstlisting}

\subsection{Software}\label{software}

\begin{itemize}
\item
  Git Bash\footnote{\url{https://git-scm.com/downloads}}
\item
  Git-Repository klonen\footnote{\url{https://github.com/ju1-eu/N-Meisterschule.git}}
\item
  Texlive (Latex)\footnote{\url{https://www.tug.org/texlive/}}
\item
  Pandoc (Dokumentenkonverter)\footnote{\url{https://pandoc.org/installing.html}}
\item
  Imagemagick (Bildbearbeitung)\footnote{\url{https://imagemagick.org/script/download.php}}
\item
  Editor Visual Studio Code\footnote{\url{https://code.visualstudio.com/}}
\item
  TeXstudio (Latexeditor)
\item
  Tablesgenerator (Latex / Markdown)\footnote{\url{https://www.tablesgenerator.com/latex_tables}}
\item
  hpi-dokumentvorlagen-latex (Hasso-Plattner-Institut (HPI)
  Potsdam)\footnote{\url{https://osm.hpi.de/theses/tipps\#dokumentvorlagen-latex}}
\item
  Zotero (Literaturverwaltung)\footnote{\url{http://www.zotero.org/}}
\item
  WordPress\footnote{\url{https://de.wordpress.org/download/}}
\item
  XAMPP Apache + Maria DB + PHP\footnote{\url{https://www.apachefriends.org/de/index.html}}
\item
  FileZilla\footnote{\url{https://filezilla-project.org/}}
\item
  VM VirtualBox\footnote{\url{https://www.virtualbox.org/}}
\item
  Ubuntu (Desktop / Server)\footnote{\url{https://ubuntu.com/download}}
\item
  WordPress-Themes\footnote{\url{https://de.wordpress.org/themes/}}
\item
  themecheck (WordPress-Themes)\footnote{\url{https://themecheck.info/}}
\item
  ghostscript Z.~B. EPS in PDF\footnote{\url{https://www.ghostscript.com/}}
\end{itemize}

\subsection{Erste Schritte}\label{erste-schritte}

\textbf{Files anpassen:}

\begin{enumerate}
\item
  \verb|scripteBash/sed.sh|

  \begin{itemize}
  \item
    codelanguage:
    \verb|HTML5, Python, Bash, C, C++, TeX|
  \item
    CMS Server Pfad: \verb|https://bw-ju.de/\#|
  \item
    Bildformat: SVG, PNG, JPG, WebP
  \end{itemize}
\item
  \verb|scripteBash/gitversionieren.sh|

  \begin{itemize}
  \item
    >>/Volumes/USB-DATEN/meineNotizen/repository/notizen-iMac<<
  \end{itemize}
\item
  \verb|projekt.sh|

  \begin{itemize}
  \item
    THEMA=>>N-Meisterschule<<
  \item
    >>/Volumes/USB-DATEN/meineNotizen/backup/notizen-iMac<<
  \item
    >>/Volumes/USB-DATEN/meineNotizen/archiv/notizen-iMac<<
  \end{itemize}
\item
  \verb|content/meta.tex|

  \begin{itemize}
  \item
    Datum, Titel, Autor
  \end{itemize}
\item
  \verb|content/titelpage.tex|

  \begin{itemize}
  \item
    >>Grafiken/logo.eps<<
  \end{itemize}
\end{enumerate}

\textbf{Markdown-Files erstellen}

\begin{enumerate}
\item
  Erstelle eine Datei >>neu.md<< im Ordner >>md/<<

  \begin{itemize}
  \item
    Bilder nach \verb|images/| kopieren
  \item
    Vektorgrafiken nach \verb|images/| kopieren
  \end{itemize}
\item
  Script ausführen: \verb|projekt.sh|
\end{enumerate}

\textbf{Terminal} öffnen

\lstset{language=Python}% C, TeX, Bash, Python 
\begin{lstlisting}[
	%caption={}, label={code:}%% anpassen
]
$ ./projekt.sh

    0) Projekt aufräumen
    1) Projekt erstellen
    2) Markdown in (tex, html5) + sed (Suchen/Ersetzen)
    3) Kapitel erstellen + Scripte ausführen
    4) Fotos optimieren (Web, Latex)
    5) www + index.html
    6) git init
    7) git status + git log
    8) Git-Version erstellen
    9) Backup + Archiv erstellen
    10) Beenden?

Eingabe Zahl >_
\end{lstlisting}

\begin{enumerate}
\setcounter{enumi}{2}
\item
  Latex-PDFs erstellen: \verb|make|
\end{enumerate}

\lstset{language=Python}% C, TeX, Bash, Python 
\begin{lstlisting}[
	%caption={}, label={code:}%% anpassen
]
$ make
$ make clean
$ make distclean
\end{lstlisting}

\begin{enumerate}
\setcounter{enumi}{3}
\item
  Repository auf Github erstellen
\end{enumerate}

\subsection{Git-Repository erstellen --
klonen}\label{git-repository-erstellen-klonen}

GitHub's Maximum File size of \textbf{50 MB}

\textbf{Repository auf Github erstellen}

\lstset{language=Python}% C, TeX, Bash, Python 
\begin{lstlisting}[
	%caption={}, label={code:}%% anpassen
]
# HTTPS oder SSH
HTTPS: https://github.com/ju1-eu/N-Meisterschule.git
SSH: git@github.com:ju1-eu/N-Meisterschule.git

# create a new repository 
echo "# README" >> README.md
# iMac Warnung 
# git config --global init.defaultBranch master
git init
git add .
git commit -m "git init"
                
# or push an existing repository 
git remote add origin https://github.com/ju1-eu/N-Meisterschule.git
git push -u origin master
# new
git push -u origin main
\end{lstlisting}

\textbf{Git-Repository klonen}

\lstset{language=Python}% C, TeX, Bash, Python 
\begin{lstlisting}[
	%caption={}, label={code:}%% anpassen
]
git clone https://github.com/ju1-eu/N-Meisterschule.git
\end{lstlisting}

\subsection{Script Beschreibung}\label{script-beschreibung}

\verb|$ ./projekt.sh|

\begin{enumerate}
\item
  Projekt erstellen

  \begin{itemize}
  \item
    Verzeichnis erstellen, wenn nicht vorhanden
  \end{itemize}
\item
  Markdown in \verb|*.tex und *.html|

  \begin{itemize}
  \item
    Markdown in Latex + HTML5 + WordPress
  \item
    sed > WordPress
  \item
    sed > Latex
  \end{itemize}
\item
  Kapitel erstellen + Scripte ausführen

  \begin{itemize}
  \item
    Alle Abbildungen >>images/<< in Markdown speichern.

    \begin{itemize}
    \item
      >>archiv/input-img.txt<<
    \end{itemize}
  \item
    Latex Kapitel erstellen.

    \begin{itemize}
    \item
      Kopiere >>texPandoc/.tex<< nach >>content/tex/<<
    \item
      >>content/tex/<< \textbf{Handarbeit\ldots{}} für optimale
      Ergebnisse!
    \item
      Kopiere >>archiv/inhalt.tex<< nach >>content/<<
    \item
      make -- Latex-PDF erstellen
    \end{itemize}
  \item
    Tabellen als PDFs in Latex einfügen. >>Tabellen/ ?<<
  \item
    Inhalt vom Projektverzeichnis.

    \begin{itemize}
    \item
      >>archiv/Projekt-Inhalt.txt<<
    \end{itemize}
  \item
    Quellcode >>code/<< in Latex speichern.

    \begin{itemize}
    \item
      >>archiv/Quellcode-files.tex<< HTML, Python, Bash, C, C++, TeX
    \end{itemize}
  \item
    Artikel aus den Ordnern erstellen

    \begin{itemize}
    \item
      >>content/tex/<<
    \item
      >>archiv/<<
    \item
      >>Tabellen/<<
    \item
      >>content/beispiele/tex/<<
    \item
      wird gespeichert in >>Artikel/<<
    \end{itemize}
  \item
    Alle Abbildungen >>images/<< in Latex speichern

    \begin{itemize}
    \item
      >>archiv/Pics-files.tex<<
    \item
      Bildgröße: \verb|width=.80\\textwidth|
    \end{itemize}
  \end{itemize}
\end{enumerate}

\begin{enumerate}
\setcounter{enumi}{3}
\item
  Fotos optimieren (Web, Latex)
\item
  www + index.html

  \begin{itemize}
  \item
    >>html/alle-pics.html<< erstellen
  \item
    >>index.html<< erstellen
  \end{itemize}
\item
  \verb|git init|
\item
  \verb|git status| +
  \verb|git log|
\item
  Git-Version erstellen

  \begin{itemize}
  \item
    \textbf{Pfade} anpassen in
    \verb|gitversionieren.sh|
  \item
    lokales Repository: master
  \item
    Github Repository: \verb|origin/master|
    \textbf{new:} \verb|origin/main|
  \item
    Sicherung Repository: backupUSB/master

    \begin{itemize}
    \item
      >>/Volumes/USB-DATEN/meineNotizen/repository/notizen-iMac<<
    \end{itemize}
  \end{itemize}
\item
  Sicherung + Archiv erstellen

  \begin{itemize}
  \item
    \textbf{Pfade} anpassen in \verb|projekt.sh|
  \item
    THEMA=>>N-Meisterschule<<
  \item
    >>/Volumes/USB-DATEN/meineNotizen/backup/notizen-iMac<<
  \item
    >>/Volumes/USB-DATEN/meineNotizen/archiv/notizen-iMac<<
  \end{itemize}
\end{enumerate}
